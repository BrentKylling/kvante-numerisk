
% Default to the notebook output style


% Inherit from the specified cell style.





    
\documentclass[11pt]{article}

    
    

    \usepackage[T1]{fontenc}
    % Nicer default font (+ math font) than Computer Modern for most use cases
    \usepackage{mathpazo}

    % Basic figure setup, for now with no caption control since it's done
    % automatically by Pandoc (which extracts ![](path) syntax from Markdown).
    \usepackage{graphicx}
    % We will generate all images so they have a width \maxwidth. This means
    % that they will get their normal width if they fit onto the page, but
    % are scaled down if they would overflow the margins.
    \makeatletter
    \def\maxwidth{\ifdim\Gin@nat@width>\linewidth\linewidth
    \else\Gin@nat@width\fi}
    \makeatother
    \let\Oldincludegraphics\includegraphics
    % Set max figure width to be 80% of text width, for now hardcoded.
    \renewcommand{\includegraphics}[1]{\Oldincludegraphics[width=.8\maxwidth]{#1}}
    % Ensure that by default, figures have no caption (until we provide a
    % proper Figure object with a Caption API and a way to capture that
    % in the conversion process - todo).
    \usepackage{caption}
    \DeclareCaptionLabelFormat{nolabel}{}
    \captionsetup{labelformat=nolabel}

    \usepackage{adjustbox} % Used to constrain images to a maximum size 
    \usepackage{xcolor} % Allow colors to be defined
    \usepackage{enumerate} % Needed for markdown enumerations to work
    \usepackage{geometry} % Used to adjust the document margins
    \usepackage{amsmath} % Equations
    \usepackage{amssymb} % Equations
    \usepackage{textcomp} % defines textquotesingle
    % Hack from http://tex.stackexchange.com/a/47451/13684:
    \AtBeginDocument{%
        \def\PYZsq{\textquotesingle}% Upright quotes in Pygmentized code
    }
    \usepackage{upquote} % Upright quotes for verbatim code
    \usepackage{eurosym} % defines \euro
    \usepackage[mathletters]{ucs} % Extended unicode (utf-8) support
    \usepackage[utf8x]{inputenc} % Allow utf-8 characters in the tex document
    \usepackage{fancyvrb} % verbatim replacement that allows latex
    \usepackage{grffile} % extends the file name processing of package graphics 
                         % to support a larger range 
    % The hyperref package gives us a pdf with properly built
    % internal navigation ('pdf bookmarks' for the table of contents,
    % internal cross-reference links, web links for URLs, etc.)
    \usepackage{hyperref}
    \usepackage{longtable} % longtable support required by pandoc >1.10
    \usepackage{booktabs}  % table support for pandoc > 1.12.2
    \usepackage[inline]{enumitem} % IRkernel/repr support (it uses the enumerate* environment)
    \usepackage[normalem]{ulem} % ulem is needed to support strikethroughs (\sout)
                                % normalem makes italics be italics, not underlines
    
    \usepackage{listings} % Used to define pretty listings for code sections [jfb]
    \usepackage{float}   


    
    
    % Colors for the hyperref package
    \definecolor{urlcolor}{rgb}{0,.145,.698}
    \definecolor{linkcolor}{rgb}{.71,0.21,0.01}
    \definecolor{citecolor}{rgb}{.12,.54,.11}

    % ANSI colors
    \definecolor{ansi-black}{HTML}{3E424D}
    \definecolor{ansi-black-intense}{HTML}{282C36}
    \definecolor{ansi-red}{HTML}{E75C58}
    \definecolor{ansi-red-intense}{HTML}{B22B31}
    \definecolor{ansi-green}{HTML}{00A250}
    \definecolor{ansi-green-intense}{HTML}{007427}
    \definecolor{ansi-yellow}{HTML}{DDB62B}
    \definecolor{ansi-yellow-intense}{HTML}{B27D12}
    \definecolor{ansi-blue}{HTML}{208FFB}
    \definecolor{ansi-blue-intense}{HTML}{0065CA}
    \definecolor{ansi-magenta}{HTML}{D160C4}
    \definecolor{ansi-magenta-intense}{HTML}{A03196}
    \definecolor{ansi-cyan}{HTML}{60C6C8}
    \definecolor{ansi-cyan-intense}{HTML}{258F8F}
    \definecolor{ansi-white}{HTML}{C5C1B4}
    \definecolor{ansi-white-intense}{HTML}{A1A6B2}

    % commands and environments needed by pandoc snippets
    % extracted from the output of `pandoc -s`
    \providecommand{\tightlist}{%
      \setlength{\itemsep}{0pt}\setlength{\parskip}{0pt}}
    \DefineVerbatimEnvironment{Highlighting}{Verbatim}{commandchars=\\\{\}}
    % Add ',fontsize=\small' for more characters per line
    \newenvironment{Shaded}{}{}
    \newcommand{\KeywordTok}[1]{\textcolor[rgb]{0.00,0.44,0.13}{\textbf{{#1}}}}
    \newcommand{\DataTypeTok}[1]{\textcolor[rgb]{0.56,0.13,0.00}{{#1}}}
    \newcommand{\DecValTok}[1]{\textcolor[rgb]{0.25,0.63,0.44}{{#1}}}
    \newcommand{\BaseNTok}[1]{\textcolor[rgb]{0.25,0.63,0.44}{{#1}}}
    \newcommand{\FloatTok}[1]{\textcolor[rgb]{0.25,0.63,0.44}{{#1}}}
    \newcommand{\CharTok}[1]{\textcolor[rgb]{0.25,0.44,0.63}{{#1}}}
    \newcommand{\StringTok}[1]{\textcolor[rgb]{0.25,0.44,0.63}{{#1}}}
    \newcommand{\CommentTok}[1]{\textcolor[rgb]{0.38,0.63,0.69}{\textit{{#1}}}}
    \newcommand{\OtherTok}[1]{\textcolor[rgb]{0.00,0.44,0.13}{{#1}}}
    \newcommand{\AlertTok}[1]{\textcolor[rgb]{1.00,0.00,0.00}{\textbf{{#1}}}}
    \newcommand{\FunctionTok}[1]{\textcolor[rgb]{0.02,0.16,0.49}{{#1}}}
    \newcommand{\RegionMarkerTok}[1]{{#1}}
    \newcommand{\ErrorTok}[1]{\textcolor[rgb]{1.00,0.00,0.00}{\textbf{{#1}}}}
    \newcommand{\NormalTok}[1]{{#1}}
    
    % Additional commands for more recent versions of Pandoc
    \newcommand{\ConstantTok}[1]{\textcolor[rgb]{0.53,0.00,0.00}{{#1}}}
    \newcommand{\SpecialCharTok}[1]{\textcolor[rgb]{0.25,0.44,0.63}{{#1}}}
    \newcommand{\VerbatimStringTok}[1]{\textcolor[rgb]{0.25,0.44,0.63}{{#1}}}
    \newcommand{\SpecialStringTok}[1]{\textcolor[rgb]{0.73,0.40,0.53}{{#1}}}
    \newcommand{\ImportTok}[1]{{#1}}
    \newcommand{\DocumentationTok}[1]{\textcolor[rgb]{0.73,0.13,0.13}{\textit{{#1}}}}
    \newcommand{\AnnotationTok}[1]{\textcolor[rgb]{0.38,0.63,0.69}{\textbf{\textit{{#1}}}}}
    \newcommand{\CommentVarTok}[1]{\textcolor[rgb]{0.38,0.63,0.69}{\textbf{\textit{{#1}}}}}
    \newcommand{\VariableTok}[1]{\textcolor[rgb]{0.10,0.09,0.49}{{#1}}}
    \newcommand{\ControlFlowTok}[1]{\textcolor[rgb]{0.00,0.44,0.13}{\textbf{{#1}}}}
    \newcommand{\OperatorTok}[1]{\textcolor[rgb]{0.40,0.40,0.40}{{#1}}}
    \newcommand{\BuiltInTok}[1]{{#1}}
    \newcommand{\ExtensionTok}[1]{{#1}}
    \newcommand{\PreprocessorTok}[1]{\textcolor[rgb]{0.74,0.48,0.00}{{#1}}}
    \newcommand{\AttributeTok}[1]{\textcolor[rgb]{0.49,0.56,0.16}{{#1}}}
    \newcommand{\InformationTok}[1]{\textcolor[rgb]{0.38,0.63,0.69}{\textbf{\textit{{#1}}}}}
    \newcommand{\WarningTok}[1]{\textcolor[rgb]{0.38,0.63,0.69}{\textbf{\textit{{#1}}}}}
    
    
    % Define a nice break command that doesn't care if a line doesn't already
    % exist.
    \def\br{\hspace*{\fill} \\* }
    % Math Jax compatability definitions
    \def\gt{>}
    \def\lt{<}
    % Document parameters
    
\title{ }

    
    
\author{J.-F. Bercher}

    

    % Pygments definitions
    
\makeatletter
\def\PY@reset{\let\PY@it=\relax \let\PY@bf=\relax%
    \let\PY@ul=\relax \let\PY@tc=\relax%
    \let\PY@bc=\relax \let\PY@ff=\relax}
\def\PY@tok#1{\csname PY@tok@#1\endcsname}
\def\PY@toks#1+{\ifx\relax#1\empty\else%
    \PY@tok{#1}\expandafter\PY@toks\fi}
\def\PY@do#1{\PY@bc{\PY@tc{\PY@ul{%
    \PY@it{\PY@bf{\PY@ff{#1}}}}}}}
\def\PY#1#2{\PY@reset\PY@toks#1+\relax+\PY@do{#2}}

\expandafter\def\csname PY@tok@w\endcsname{\def\PY@tc##1{\textcolor[rgb]{0.73,0.73,0.73}{##1}}}
\expandafter\def\csname PY@tok@c\endcsname{\let\PY@it=\textit\def\PY@tc##1{\textcolor[rgb]{0.25,0.50,0.50}{##1}}}
\expandafter\def\csname PY@tok@cp\endcsname{\def\PY@tc##1{\textcolor[rgb]{0.74,0.48,0.00}{##1}}}
\expandafter\def\csname PY@tok@k\endcsname{\let\PY@bf=\textbf\def\PY@tc##1{\textcolor[rgb]{0.00,0.50,0.00}{##1}}}
\expandafter\def\csname PY@tok@kp\endcsname{\def\PY@tc##1{\textcolor[rgb]{0.00,0.50,0.00}{##1}}}
\expandafter\def\csname PY@tok@kt\endcsname{\def\PY@tc##1{\textcolor[rgb]{0.69,0.00,0.25}{##1}}}
\expandafter\def\csname PY@tok@o\endcsname{\def\PY@tc##1{\textcolor[rgb]{0.40,0.40,0.40}{##1}}}
\expandafter\def\csname PY@tok@ow\endcsname{\let\PY@bf=\textbf\def\PY@tc##1{\textcolor[rgb]{0.67,0.13,1.00}{##1}}}
\expandafter\def\csname PY@tok@nb\endcsname{\def\PY@tc##1{\textcolor[rgb]{0.00,0.50,0.00}{##1}}}
\expandafter\def\csname PY@tok@nf\endcsname{\def\PY@tc##1{\textcolor[rgb]{0.00,0.00,1.00}{##1}}}
\expandafter\def\csname PY@tok@nc\endcsname{\let\PY@bf=\textbf\def\PY@tc##1{\textcolor[rgb]{0.00,0.00,1.00}{##1}}}
\expandafter\def\csname PY@tok@nn\endcsname{\let\PY@bf=\textbf\def\PY@tc##1{\textcolor[rgb]{0.00,0.00,1.00}{##1}}}
\expandafter\def\csname PY@tok@ne\endcsname{\let\PY@bf=\textbf\def\PY@tc##1{\textcolor[rgb]{0.82,0.25,0.23}{##1}}}
\expandafter\def\csname PY@tok@nv\endcsname{\def\PY@tc##1{\textcolor[rgb]{0.10,0.09,0.49}{##1}}}
\expandafter\def\csname PY@tok@no\endcsname{\def\PY@tc##1{\textcolor[rgb]{0.53,0.00,0.00}{##1}}}
\expandafter\def\csname PY@tok@nl\endcsname{\def\PY@tc##1{\textcolor[rgb]{0.63,0.63,0.00}{##1}}}
\expandafter\def\csname PY@tok@ni\endcsname{\let\PY@bf=\textbf\def\PY@tc##1{\textcolor[rgb]{0.60,0.60,0.60}{##1}}}
\expandafter\def\csname PY@tok@na\endcsname{\def\PY@tc##1{\textcolor[rgb]{0.49,0.56,0.16}{##1}}}
\expandafter\def\csname PY@tok@nt\endcsname{\let\PY@bf=\textbf\def\PY@tc##1{\textcolor[rgb]{0.00,0.50,0.00}{##1}}}
\expandafter\def\csname PY@tok@nd\endcsname{\def\PY@tc##1{\textcolor[rgb]{0.67,0.13,1.00}{##1}}}
\expandafter\def\csname PY@tok@s\endcsname{\def\PY@tc##1{\textcolor[rgb]{0.73,0.13,0.13}{##1}}}
\expandafter\def\csname PY@tok@sd\endcsname{\let\PY@it=\textit\def\PY@tc##1{\textcolor[rgb]{0.73,0.13,0.13}{##1}}}
\expandafter\def\csname PY@tok@si\endcsname{\let\PY@bf=\textbf\def\PY@tc##1{\textcolor[rgb]{0.73,0.40,0.53}{##1}}}
\expandafter\def\csname PY@tok@se\endcsname{\let\PY@bf=\textbf\def\PY@tc##1{\textcolor[rgb]{0.73,0.40,0.13}{##1}}}
\expandafter\def\csname PY@tok@sr\endcsname{\def\PY@tc##1{\textcolor[rgb]{0.73,0.40,0.53}{##1}}}
\expandafter\def\csname PY@tok@ss\endcsname{\def\PY@tc##1{\textcolor[rgb]{0.10,0.09,0.49}{##1}}}
\expandafter\def\csname PY@tok@sx\endcsname{\def\PY@tc##1{\textcolor[rgb]{0.00,0.50,0.00}{##1}}}
\expandafter\def\csname PY@tok@m\endcsname{\def\PY@tc##1{\textcolor[rgb]{0.40,0.40,0.40}{##1}}}
\expandafter\def\csname PY@tok@gh\endcsname{\let\PY@bf=\textbf\def\PY@tc##1{\textcolor[rgb]{0.00,0.00,0.50}{##1}}}
\expandafter\def\csname PY@tok@gu\endcsname{\let\PY@bf=\textbf\def\PY@tc##1{\textcolor[rgb]{0.50,0.00,0.50}{##1}}}
\expandafter\def\csname PY@tok@gd\endcsname{\def\PY@tc##1{\textcolor[rgb]{0.63,0.00,0.00}{##1}}}
\expandafter\def\csname PY@tok@gi\endcsname{\def\PY@tc##1{\textcolor[rgb]{0.00,0.63,0.00}{##1}}}
\expandafter\def\csname PY@tok@gr\endcsname{\def\PY@tc##1{\textcolor[rgb]{1.00,0.00,0.00}{##1}}}
\expandafter\def\csname PY@tok@ge\endcsname{\let\PY@it=\textit}
\expandafter\def\csname PY@tok@gs\endcsname{\let\PY@bf=\textbf}
\expandafter\def\csname PY@tok@gp\endcsname{\let\PY@bf=\textbf\def\PY@tc##1{\textcolor[rgb]{0.00,0.00,0.50}{##1}}}
\expandafter\def\csname PY@tok@go\endcsname{\def\PY@tc##1{\textcolor[rgb]{0.53,0.53,0.53}{##1}}}
\expandafter\def\csname PY@tok@gt\endcsname{\def\PY@tc##1{\textcolor[rgb]{0.00,0.27,0.87}{##1}}}
\expandafter\def\csname PY@tok@err\endcsname{\def\PY@bc##1{\setlength{\fboxsep}{0pt}\fcolorbox[rgb]{1.00,0.00,0.00}{1,1,1}{\strut ##1}}}
\expandafter\def\csname PY@tok@kc\endcsname{\let\PY@bf=\textbf\def\PY@tc##1{\textcolor[rgb]{0.00,0.50,0.00}{##1}}}
\expandafter\def\csname PY@tok@kd\endcsname{\let\PY@bf=\textbf\def\PY@tc##1{\textcolor[rgb]{0.00,0.50,0.00}{##1}}}
\expandafter\def\csname PY@tok@kn\endcsname{\let\PY@bf=\textbf\def\PY@tc##1{\textcolor[rgb]{0.00,0.50,0.00}{##1}}}
\expandafter\def\csname PY@tok@kr\endcsname{\let\PY@bf=\textbf\def\PY@tc##1{\textcolor[rgb]{0.00,0.50,0.00}{##1}}}
\expandafter\def\csname PY@tok@bp\endcsname{\def\PY@tc##1{\textcolor[rgb]{0.00,0.50,0.00}{##1}}}
\expandafter\def\csname PY@tok@fm\endcsname{\def\PY@tc##1{\textcolor[rgb]{0.00,0.00,1.00}{##1}}}
\expandafter\def\csname PY@tok@vc\endcsname{\def\PY@tc##1{\textcolor[rgb]{0.10,0.09,0.49}{##1}}}
\expandafter\def\csname PY@tok@vg\endcsname{\def\PY@tc##1{\textcolor[rgb]{0.10,0.09,0.49}{##1}}}
\expandafter\def\csname PY@tok@vi\endcsname{\def\PY@tc##1{\textcolor[rgb]{0.10,0.09,0.49}{##1}}}
\expandafter\def\csname PY@tok@vm\endcsname{\def\PY@tc##1{\textcolor[rgb]{0.10,0.09,0.49}{##1}}}
\expandafter\def\csname PY@tok@sa\endcsname{\def\PY@tc##1{\textcolor[rgb]{0.73,0.13,0.13}{##1}}}
\expandafter\def\csname PY@tok@sb\endcsname{\def\PY@tc##1{\textcolor[rgb]{0.73,0.13,0.13}{##1}}}
\expandafter\def\csname PY@tok@sc\endcsname{\def\PY@tc##1{\textcolor[rgb]{0.73,0.13,0.13}{##1}}}
\expandafter\def\csname PY@tok@dl\endcsname{\def\PY@tc##1{\textcolor[rgb]{0.73,0.13,0.13}{##1}}}
\expandafter\def\csname PY@tok@s2\endcsname{\def\PY@tc##1{\textcolor[rgb]{0.73,0.13,0.13}{##1}}}
\expandafter\def\csname PY@tok@sh\endcsname{\def\PY@tc##1{\textcolor[rgb]{0.73,0.13,0.13}{##1}}}
\expandafter\def\csname PY@tok@s1\endcsname{\def\PY@tc##1{\textcolor[rgb]{0.73,0.13,0.13}{##1}}}
\expandafter\def\csname PY@tok@mb\endcsname{\def\PY@tc##1{\textcolor[rgb]{0.40,0.40,0.40}{##1}}}
\expandafter\def\csname PY@tok@mf\endcsname{\def\PY@tc##1{\textcolor[rgb]{0.40,0.40,0.40}{##1}}}
\expandafter\def\csname PY@tok@mh\endcsname{\def\PY@tc##1{\textcolor[rgb]{0.40,0.40,0.40}{##1}}}
\expandafter\def\csname PY@tok@mi\endcsname{\def\PY@tc##1{\textcolor[rgb]{0.40,0.40,0.40}{##1}}}
\expandafter\def\csname PY@tok@il\endcsname{\def\PY@tc##1{\textcolor[rgb]{0.40,0.40,0.40}{##1}}}
\expandafter\def\csname PY@tok@mo\endcsname{\def\PY@tc##1{\textcolor[rgb]{0.40,0.40,0.40}{##1}}}
\expandafter\def\csname PY@tok@ch\endcsname{\let\PY@it=\textit\def\PY@tc##1{\textcolor[rgb]{0.25,0.50,0.50}{##1}}}
\expandafter\def\csname PY@tok@cm\endcsname{\let\PY@it=\textit\def\PY@tc##1{\textcolor[rgb]{0.25,0.50,0.50}{##1}}}
\expandafter\def\csname PY@tok@cpf\endcsname{\let\PY@it=\textit\def\PY@tc##1{\textcolor[rgb]{0.25,0.50,0.50}{##1}}}
\expandafter\def\csname PY@tok@c1\endcsname{\let\PY@it=\textit\def\PY@tc##1{\textcolor[rgb]{0.25,0.50,0.50}{##1}}}
\expandafter\def\csname PY@tok@cs\endcsname{\let\PY@it=\textit\def\PY@tc##1{\textcolor[rgb]{0.25,0.50,0.50}{##1}}}

\def\PYZbs{\char`\\}
\def\PYZus{\char`\_}
\def\PYZob{\char`\{}
\def\PYZcb{\char`\}}
\def\PYZca{\char`\^}
\def\PYZam{\char`\&}
\def\PYZlt{\char`\<}
\def\PYZgt{\char`\>}
\def\PYZsh{\char`\#}
\def\PYZpc{\char`\%}
\def\PYZdl{\char`\$}
\def\PYZhy{\char`\-}
\def\PYZsq{\char`\'}
\def\PYZdq{\char`\"}
\def\PYZti{\char`\~}
% for compatibility with earlier versions
\def\PYZat{@}
\def\PYZlb{[}
\def\PYZrb{]}
\makeatother



    
    % Prevent overflowing lines due to hard-to-break entities
    \sloppy
    % Setup hyperref package
    \hypersetup{
    breaklinks=true, % so long urls are correctly broken across lines
	pdftitle={\@title},
	pdfauthor={\@author},
	colorlinks=true, % color links 
	breaklinks=true, % enable to break long links
	urlcolor= blue,  % color for external links
	linkcolor= blue, % color for external links
	citecolor=blue,  
	bookmarksopen=false,
	pdftoolbar=false,
	pdfmenubar=false,
%      hidelinks
      }
    % Slightly bigger margins than the latex defaults
    \geometry{verbose,tmargin=1in,bmargin=1in,lmargin=1in,rmargin=1in}
    %listings configuration

\definecolor{mygreen}{rgb}{0,0.6,0}
\definecolor{mygray}{rgb}{0.5,0.5,0.5}
\definecolor{mymauve}{rgb}{0.58,0,0.82}
	\lstset{
language=Python,
commentstyle=\color{mygreen},
keywordstyle=\color{blue},
stringstyle=\color{mymauve},
xleftmargin= 1cm,
xrightmargin= 1cm,
showstringspaces=false,
	   breaklines=true,
	   texcl=false,
%	   basicstyle=\ttfamily,
frame=single,
frameround=tttt,
framesep=10pt,
%framexleftmargin=10pt,
%framexrightmargin =10pt,
%frameshape={RYRYNYYYY}{yny}{yny}{RYRYNYYYY} 
        inputencoding=utf8,
        extendedchars=true,
        literate=%
        {é}{{\'{e}}}1
        {è}{{\`{e}}}1
        {ê}{{\^{e}}}1
        {ë}{{\¨{e}}}1
        {É}{{\'{E}}}1
        {Ê}{{\^{E}}}1
        {û}{{\^{u}}}1
        {ù}{{\`{u}}}1
        {à}{{\`{a}}}1
        {ç}{{\c{c}}}1
        {Ç}{{\c{C}}}1
        {î}{{\^{i}}}1
        {Î}{{\^{I}}}1
}


%\usepackage{foo}

    \begin{document}
    
    
    \maketitle
    
\tableofcontents

    


    %
\begin{lstlisting}
from numpy import *
import matplotlib.pyplot as plt
from IPython.display import HTML, display
from matplotlib import animation
import scipy.sparse as sp
import warnings

%matplotlib inline
plt.rcParams['animation.html'] = 'html5'
warnings.filterwarnings('ignore')


####################CONST####################
hbar = 1.05E-34
m = 9.11E-31    # elektronmasse
V0 = 1.6E-19
Ntot = 1000     # Intervallbredde
N = 100         # Parabolbredde
dx = 1.0E-10    # 1Å
harmonic = True
##############################################

#################VARIABLE#####################
sigma = N / 4 * dx
p0 = 3E-25
k0 = p0 / hbar
omega = sqrt(k0 / m)
meanE = p0 ** 2 / (2 * m)
alpha = 1 / (2 * sigma ** 2)
##############################################

def V(x):
    if not harmonic:         # Fri partikkel
        return zeros_like(x)
    else:                    # Harmonisk
        V = V0 * ((x - x0) / (N * dx)) ** 2  # N=x/dx
        mask = ((x0 - N * dx) < x) & (x < (x0 + N * dx))
        return where(mask, V, V0)
    
def animator(time=200,figsize=(7,4)):
    def init():
        line.set_data([], [])
        marker1.set_data([], [])
        marker2.set_data([], [])
        marker3.set_data([], [])
        return line, marker1, marker2, marker3,

    def animate(i):
        t = tArr[i]
        line.set_data(x, BigPsiSquared(t))
        marker1.set_data(meanX(t) + sigmaX(t), 0)
        marker2.set_data(meanX(t) - sigmaX(t), 0)
        marker3.set_data(meanX(t), 0)
        return line, marker1, marker2, marker3,

    tArr = arange(time) * 3E-15

    fig = plt.figure(figsize=figsize)
    ymax = 1.7E8 
    ax = plt.axes(xlim=(0, Ntot * dx), ylim=(0, ymax))
    line, = ax.plot([], [], lw=1,label=r'$\left| \Psi \right|^2$')
    marker1, = ax.plot([], [], 'bo',label=r'$\Delta x_t$')
    marker2, = ax.plot([], [], 'bo')
    marker3, = ax.plot([], [], 'co',label=r'$\langle x \rangle_t$')
    plt.xlabel('$x$ (m)')

    if harmonic:
        plt.xlim(0.3E-7,0.7E-7)
        b0 = x[abs(V(x) * 10 ** 27 - ones_like(x) * meanE * 10 ** 27) < 0.005E8]
        print(b0)
        
        plt.plot(x, ones_like(x) * meanE * 10 ** 27,label='$E=p_0^2/2m$')
        plt.plot(x, V(x) * 10 ** 27,label='$V$')
        plt.plot(x0, 0, 'ko', label='$x_0$')
        plt.plot(b0, zeros_like(b0), 'ro',label='$b_0$')
    
    plt.legend(loc='upper left', borderaxespad=1.5,borderpad=0.1)
    anim=animation.FuncAnimation(fig, animate, init_func=init, repeat=True, frames=len(tArr), interval=64, blit=True)
    plt.close(fig)
    display(HTML(anim.to_html5_video()))
\end{lstlisting}
    \hypertarget{innledning}{%
\section{Innledning}\label{innledning}}

Et grunnleggende postulat i kvantemekanikken sier at tilstanden til et
system er fullstendig beskrevet av systemets bølgefunksjon \(\Psi\),
hvis løsning kommer av Schrödingerlikningen
\begin{equation}\label{eq:SL}
    i\hbar\frac{\partial\Psi}{\partial t}=\hat{H}\Psi
\end{equation} hvor \(\hat{H}=-\frac{\hbar^2}{2m}\nabla^2+V\) er
systemets Hamiltonoperator.

Ved Bornes sannsynlighetstolkning vil en kvantemekanisk partikkel kunne
beskrives i henhold til en sannsynlighetsfordeling gitt ved
\(\left|\Psi\right|^2\). For tidsuavhengige potensialer \(V=V(\vec{r})\)
vil \eqref{eq:SL} kunne løses ved seperasjon av variable, slik at
bølgefunksjonen blir på formen \begin{equation}\label{eq:Psi}
    \Psi(\vec{r},t)=\psi(\vec{r})e^{-iEt/\hbar}
\end{equation} der \(\psi\) er løsningen av den tidsuavhengige
Schrödingerlikningen \begin{equation}
    \label{eq:TUSL}
    \hat{H}\psi=E\psi
\end{equation} Dette er en egenverdilikning som kvantiserer systemets
tilstander i et fullstendig egenfunksjonsett \(\left{ \psi_j \right}\)
med tilhørende spekter \(\left{ E_j \right}\). Vi skal utelukkende
studere éndimensjonale potensialer som resulterer i diskrete utgaver av
nevnte mengder. Bølgefunksjonen \eqref{eq:Psi} beskriver en stasjonær
tilstand der sannsynlighetsfordelingen ikke avhenger av tiden. Siden
\(\hat{H}\) er en hermitesk operator vil \(\left{ \psi_j \right}\) danne
en ortogonal basis for Hilbertrommet \(\mathcal{H}\), slik at
\begin{equation}\label{eq:psiDotpsi}
    \langle \psi_i,\psi_j \rangle=\int{\psi_i^* \psi_j} \space dx=\delta_{ij}
\end{equation} for normerte \(\psi\). Dette innebærer videre at en
generell bølgefunksjon \(\Psi \in \mathcal{H}\) kan utvikles som

\begin{equation}\label{eq:expansion}
    \Psi=\sum_{j} c_j \psi_j e^{-iE_jt/\hbar}
\end{equation}

med resulterende tidsavhengig sannsynlighetsfordeling.\\
Utviklingskoeffisientene finnes ved

\begin{equation}\label{eq:expansionCoeff}
    c_j=\langle \psi_j,\Psi(x,0) \rangle
\end{equation}

og resulterer i
\(\langle \Psi,\Psi \rangle=\sum_j \left| c_j \right|^2=1 \space \forall t \space\)
når \(\left{ \psi_j \right}\) er ortonormale.

I denne øvingen ser vi nærmere på tidsutviklingen til bølgepakken hvis
starttilstand \begin{equation}\label{eq:Gaussian} 
    \Psi(x,0)=(2\pi\sigma^2)^{-1/4}e^{-x^2/4\sigma^2}e^{ip_0x/\hbar}
\end{equation}

er en overlagring av de-Broglie-bølger som beskriver
sannsynlighetsfordelingen rundt en tilnærmet lokalisert partikkel. Vi
skal visualisere \(\left | \Psi \right|^2\) i henholdsvis et harmonisk-
og konstant potensiale

\begin{equation}
  \label{eq:potensial} 
  V(x) =
  \begin{cases}
  0 & \text{Fri partikkel} \\
  V_0(x-x_0)^2 & \text{Harmonisk partikkel} \\
  \end{cases}
\end{equation}

ved å numerisk bestemme egenfunksjonene som oppfyller likning
\eqref{eq:TUSL}, og siden utvikle \(\Psi\) i henhold til
\eqref{eq:expansion}. Med denne representasjonen kan forventningsverdien
av en observerbar størrelse \(F\) beregnes som
\begin{equation}\label{eq:expectation}
    \langle F \rangle=\int\Psi^{*}\hat{F} \space \Psi \space dx
\end{equation} for enhvert tid.
%
\begin{lstlisting}
x = arange(Ntot) * dx
x0 = x[len(x) // 2] if len(x) % 2 != 0 else (x[len(x) // 2 - 1] + x[len(x) // 2]) * (1 / 2)
x=x.reshape((Ntot,1))

plt.figure(figsize=(16,4))
plt.plot(x, zeros_like(x), color='k',lw=1)
plt.plot(x0, 0, 'bo', label='$x_0$')
plt.plot(x, V(x), label='Harmonisk partikkel')
plt.legend(loc='best')
plt.xlabel('$x$ (m)')
plt.ylabel('$V$ ( J )')
plt.show()
\end{lstlisting}%
    
\begin{figure}
    \begin{center}
        \adjustimage{max size={0.7\linewidth}{0.3\paperheight}}{test_files/test_2_0.png}
    \end{center}
\end{figure}
    
    \hypertarget{numerisk-tilnuxe6rming}{%
\section{Numerisk tilnærming}\label{numerisk-tilnuxe6rming}}

Rammene for oppgaven legges ved å diskretisere x-aksen i \(N_{tot}\)
punkter med innbyrdes avstand \(\Delta x\). Vi lar \(x_0\) betegne
midtpunktet på x-aksen. Sentrert rundt \(x_0\) oppretter vi et område av
bredde \(N\) hvor det harmoniske potensialet i likning
\eqref[eq:potential]\} er definert. For resten av det diskretiserte
området defineres \(V(x)=V_0\) når potensialet er harmonisk, og ellers
lar vi hele det diskretiserte området ha \(V(x)=0\) når vi betrakter en
fri partikkel. Utenfor det diskretiserte området antar vi
\(V(x)=\infty\) slik at bølgefunksjonen følgelig må gå mot null ved
endepunktene. Ved å tilnærme \(\psi^{\prime \prime}\) med
sentraldifferanser kan \eqref{eq:TUSL} modelleres ved
\begin{equation}\label{eq:NumTUSL}
   \mathbb{H}\vec{\psi}=E\vec{\psi}
\end{equation}

der den tridiagonale matrisen
\(\mathbb{H}\in \mathbb{R}^{N_{tot} \times N_{tot}}\) har elementer

\begin{equation}
  \label{eq:matrixElem} 
  \begin{cases}
  \mathbb{H}_{nn}&= \frac{\hbar^2}{m(\Delta x)^2}+V(x_n) \\
  \mathbb{H}_{n,n\pm1}&= -\frac{\hbar^2}{2m(\Delta x)^2} \\
  \end{cases}
\end{equation}

og \(\vec{\psi}\in \mathbb{R}^{N_{tot}\times 1}\) er en egenvektor som
svarer til egenfunksjonen \(\psi\) evaluert for den diskretiserte
x-aksen \(x_n=n\Delta x \;\;\;\; n=1,..,N_{tot}\).

Ved å bruke \texttt{np.linalg.eigh(} \(\mathbb{H}\) \texttt{)} fås
\(N_{tot}\) egenvektorer \(\vec{\psi}^{(j)}\) med tilhørende egenverdier
\(E_j\) som tilfredsstiller likning \eqref{eq:NumTUSL}. Normeringen
gjøres analogt til likning \eqref{eq:psiDotpsi} ved

\begin{equation}\label{eq:psiDotpsiNum}
    \sum_{n=1}^{N_{tot}} \left| \space \psi_n^{(j)} \right|^2 \Delta x =1 \;\;\;\;  j=1,..,N_{tot}
\end{equation}

der summen går over komponentene av den aktuelle egenvektoren. Vektorene
tilsvarer med dette egenfunksjonsettet fra likning \eqref{eq:TUSL}, og
kan dermed brukes til å utvikle starttilstanden \eqref{eq:Gaussian} i
henhold til likning \eqref{eq:expansion} gitt at likning
\eqref{eq:expansionCoeff} tilnærmes ved
\begin{equation}\label{eq:NumExpansionCoeff}
    c^{(j)}=\sum_{n=1}^{N_{tot}} \psi_n^{(j)*} \Psi(x_n,0)\Delta x  \;\;\;\;  j=1,..,N_{tot}
\end{equation}
%
\begin{lstlisting}
def getBigPsiSquared(p0=p0, sigma=sigma, harmonic=harmonic):
    k0 = p0 / hbar
    omega = sqrt(k0 / m)
    meanE = p0 ** 2 / (2 * m)
    alpha = 1 / (2 * sigma ** 2)
    globals().update(p0=p0, sigma=sigma, harmonic=harmonic, k0=k0, omega=omega, meanE=meanE, alpha=alpha)

    subsupDiag = -ones(Ntot - 1) * hbar ** 2 / (2 * m * dx ** 2)
    diagon = ones(Ntot) * hbar ** 2 / (m * dx ** 2) + V(x.T)
    H = array([diagon, subsupDiag, subsupDiag]); offsets = array([0, -1, 1])
    H = sp.diags(H, offsets, shape=(Ntot, Ntot))
    E, Psi = linalg.eigh(H.todense())  
    E, Psi = asarray(E).reshape(Ntot, 1), asarray(Psi, dtype=complex128)
    quadPsi = linalg.norm(Psi, 2, 0) ** 2 * dx
    normFactor = quadPsi[0]  
    Psi /= sqrt(normFactor)  
    PsiConj = conj(Psi)

    coefficient = (2 * pi * sigma ** 2) ** (-0.25)
    envelope = exp(-(x - x0) ** 2 / (4 * sigma ** 2))
    planewave = exp(1j * k0 * x)
    Psi0 = (coefficient * envelope * planewave)
    c = (PsiConj.T @ Psi0 * dx)

    BigPsi = lambda t: Psi @ (c * (exp(-1j * E * t / hbar)))
    BigPsiSquared = lambda t: abs(BigPsi(t)) ** 2  
    return BigPsiSquared

meanX=lambda t: (x.T)@BigPsiSquared(t)*dx
meanX2=lambda t: (x.T)**2@BigPsiSquared(t)*dx
sigmaX=lambda t: sqrt(meanX2(t)-meanX(t)**2)
analyticSigmaX=lambda t: sqrt(sigma**2+hbar**2*t**2/(4*m**2*sigma**2))
meanP= lambda t: (p0 + 1j * hbar * alpha * x.T)@BigPsiSquared(t)*dx
#meanP2= lambda t: ((p0 + 1j * hbar * alpha * x.T) ** 2 + hbar ** 2 * alpha)@BigPsiSquared(t)*dx
meanP2= lambda t: ((p0 + 1j * hbar * alpha * x.T) ** 2 + hbar ** 2 * alpha)@BigPsiSquared(t)*dx
sigmaP=lambda t: sqrt(meanP2(t)-meanP(t)**2)
heisenProd=lambda t: sigmaP(t)*sigmaX(t)
\end{lstlisting}
    \hypertarget{oppgaver}{%
\section{Oppgaver}\label{oppgaver}}

Vi ønsker

Fri partikkel: (N=100, sigma=N/4) * fordeling har praktisk talt
analytisk usikkerhet i posisjon for 1E-12 * litt ``frysninger'' som
kommer når numerisk usikkerhet overstiger analytisk usikkerhet *
Usikkerheten i pos minker når partikkelen treffer veggene. Dette som gir
periodiske dupp i numerisk sigma x. Bredden på dupp henger sammen med
bredden på fordeling. Jo bredere/flatere fordeling, jo større dupp over
lengre tid ettersom fordelingen bruker lengre tid ved veggen og i større
grad presses sammen ved veggen ()

\begin{itemize}
\item
  Rimelig god overenstemmelse for 1E-21, fordeling flater ut under
  traversering, ingen ``frysninger'' og numerisk usikkerhet overstiger
  ikke analytisk usikkerhet
\item
  Grad av utflatning reflekteres i hvor fort usikkerheten i posisjon
  stiger.
\item
  sigma x stiger raskere for mindre initiell sigma
\item
  for fiksert p0 vil en større initiell sigma gi bedre tilnærmelse
  mellom numerisk og analytisk usikkerhet under traversering, men større
  avvik når partikkelen treffer veggen siden ``duppene'' blir større.
  Fordelingen synker mindre sammen for større initiell sigma.
\item
  hvordan svarer flere verdier av p0 til riktige uttrykk for sigma x
\item
  for fiksert
\end{itemize}

harmonisk potensiale: (N=100, sigma=N/4) * tyngepunktet varierer
harmonisk * når sigma p minker så øker sigma x og omvendt *
sannsynlighetsfordelingen ``slarker minst'' når sigmaene varierer
stabilt slik at heisenprod er periodisk * for en verdi av p0=3.6E-25
slik at usikkerhetene slår hverandre ut (og konvergerer til hbar/2)
slarker fortsatt sannsynlighetsfordelingen endel.
%
\begin{lstlisting}
BigPsiSquared = \
getBigPsiSquared(p0=1E-12,harmonic=False)

tArr=arange(100)*3E-15
plt.figure(figsize=(7,4))
plt.title('p0: '+str(p0))
plt.plot(tArr,sigmaX(tArr).reshape(tArr.shape),\
         label='$\sigma_{x}$ (numerisk)')
plt.plot(tArr,analyticSigmaX(tArr).reshape(tArr.shape),\
         label='$\sigma_{x}$ (analytisk)')
plt.legend(loc='best')
plt.xlabel('$t$ (s)')
plt.show()

animator();
\end{lstlisting}%
    
\begin{figure}
    \begin{center}
        \adjustimage{max size={0.7\linewidth}{0.3\paperheight}}{test_files/test_6_0.png}
    \end{center}
\end{figure}
    
    
    

    %
\begin{lstlisting}
BigPsiSquared = \
getBigPsiSquared(p0=0.9E-21,harmonic=False)

tArr=arange(100)*3E-15
plt.figure(figsize=(7,4))
plt.title('p0: '+str(p0))
plt.plot(tArr,sigmaX(tArr).reshape(tArr.shape),\
         label='$\sigma_{x}$ (numerisk)')
plt.plot(tArr,analyticSigmaX(tArr).reshape(tArr.shape),\
         label='$\sigma_{x}$ (analytisk)')
plt.legend(loc='best')
plt.xlabel('$t$ (s)')
plt.show()

animator();
\end{lstlisting}%
    
\begin{figure}
    \begin{center}
        \adjustimage{max size={0.7\linewidth}{0.3\paperheight}}{test_files/test_7_0.png}
    \end{center}
\end{figure}
    
    
    

    %
\begin{lstlisting}
BigPsiSquared = \
getBigPsiSquared(p0=0.9E-21,sigma = N / 10 * dx,harmonic=False)

tArr=arange(100)*3E-15
plt.figure(figsize=(7,4))
plt.title('p0: '+str(p0))
plt.plot(tArr,sigmaX(tArr).reshape(tArr.shape),\
         label='$\sigma_{x}$ (numerisk)')
plt.plot(tArr,analyticSigmaX(tArr).reshape(tArr.shape),\
         label='$\sigma_{x}$ (analytisk)')
plt.legend(loc='best')
plt.xlabel('$t$ (s)')
plt.show()

animator();
\end{lstlisting}%
    
\begin{figure}
    \begin{center}
        \adjustimage{max size={0.7\linewidth}{0.3\paperheight}}{test_files/test_8_0.png}
    \end{center}
\end{figure}
    
    
    

    %
\begin{lstlisting}
BigPsiSquared =\
getBigPsiSquared(p0=3E-25,sigma = N / 5 * dx,\
                 harmonic=True)

tArr=arange(200)*3E-15

fig, axArr =\
plt.subplots(1, 2, sharex=True,figsize=(7,4))

plt.suptitle('p0: '+str(p0))

axArr[0].plot(tArr,sigmaX(tArr).reshape(tArr.shape),\
              label='$\sigma_{x}$')
axArr[0].legend(loc='best')
axArr[0].set_xlabel('$t$')
axArr[1].plot(tArr,sigmaP(tArr).reshape(tArr.shape),\
              label='$\sigma_p$')
axArr[1].legend(loc='best')
axArr[1].set_xlabel('$t$')

plt.figure(figsize=(7,4))
plt.title('p0: '+str(p0))
plt.plot(tArr,heisenProd(tArr).reshape(tArr.shape),\
         label='$\sigma_p\sigma_x$')
plt.plot(tArr,ones_like(tArr)*hbar/2, label="$\hbar/2$")
plt.legend(loc='best')
plt.xlabel("$t$")
plt.show()

animator()
\end{lstlisting}%
    
\begin{figure}
    \begin{center}
        \adjustimage{max size={0.7\linewidth}{0.3\paperheight}}{test_files/test_9_0.png}
    \end{center}
\end{figure}
    
    
\begin{figure}
    \begin{center}
        \adjustimage{max size={0.7\linewidth}{0.3\paperheight}}{test_files/test_9_1.png}
    \end{center}
\end{figure}
    
    \begin{Verbatim}[commandchars=\\\{\}]
[  4.44000000e-08   5.55000000e-08]

    \end{Verbatim}

    
    

    %
\begin{lstlisting}
BigPsiSquared =\
getBigPsiSquared(p0=3.6E-25, sigma = N/4* dx,\
                 harmonic=True)

tArr=arange(200)*3E-15

fig, axArr =\
plt.subplots(1, 2, sharex=True,figsize=(7,4))

plt.suptitle('p0: '+str(p0))

axArr[0].plot(tArr,sigmaX(tArr).reshape(tArr.shape),\
              label='$\sigma_{x}$')
axArr[0].legend(loc='best')
axArr[0].set_xlabel('$t$')
axArr[1].plot(tArr,sigmaP(tArr).reshape(tArr.shape),\
              label='$\sigma_p$')
axArr[1].legend(loc='best')
axArr[1].set_xlabel('$t$')

plt.figure(figsize=(7,4))
plt.title('p0: '+str(p0))
plt.plot(tArr,heisenProd(tArr).reshape(tArr.shape),\
         label='$\sigma_p\sigma_x$')
plt.plot(tArr,ones_like(tArr)*hbar/2, label="$\hbar/2$")
plt.legend(loc='best')
plt.xlabel("$t$")
plt.show()

animator()
\end{lstlisting}%
    
\begin{figure}
    \begin{center}
        \adjustimage{max size={0.7\linewidth}{0.3\paperheight}}{test_files/test_10_0.png}
    \end{center}
\end{figure}
    
    
\begin{figure}
    \begin{center}
        \adjustimage{max size={0.7\linewidth}{0.3\paperheight}}{test_files/test_10_1.png}
    \end{center}
\end{figure}
    
    \begin{Verbatim}[commandchars=\\\{\}]
[  4.33000000e-08   5.66000000e-08]

    \end{Verbatim}

    
    

    

    % Add a bibliography block to the postdoc
    
    
%\bibliographystyle{ieetran}
%\bibliography{Thesis}

    
    \end{document}
